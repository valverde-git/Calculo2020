%!TEX root = ../Calculo20.tex
%!TEX TS-program = pdflatex
%!TEX encoding = UTF-8 Unicode
%
% Para pensar en el futuro:
% Esquemas de series implicando pol, exp, fact, log
% Notación O(f(n))
% ¿qué hay que demostrar y que no?
%
\chapter{Aplicaciones}

En este tema vamos a recoger las aplicaciones y problemas que se han eliminado de los apuntes del curso pasado y que serán susceptibles de ser preguntados en examen como ejercicios auto-contenidos donde se facilitará al alumno el contenido, quizás algún ejemplo y se le pedirá que resuelva un ejercicio similar al ejemplo y de aplicación del contenido.


\section{Aplicaciones del Tema 1}

\subsection{Polinomios de grado cuatro.}

Vamos a ver unos ejemplos en los que mostramos cómo es posible obtener la factorización de polinomios de grado cuatro.

Naturalmente, primero buscaremos raíces entre los divisores del término independiente y en caso de que no tenga este tipo de ceros, usaremos los métodos explicados en los siguientes ejemplos.

Por otra parte, en estos ejemplos trabajamos con polinomios en los cuales el coeficiente del término de grado 3 es nulo. Porque, para factorizar un polinomio general, primero tendríamos que cambiar el centro del polinomio.
%En las relaciones de ejercicios factorizaremos polinomios generales de grado cuatro.

\begin{ejemplo}
Vamos a factorizar el polinomio $x^4-3x^2-6x-2$ en $\mathbb{R}$ y en $\mathbb{C}$ utilizando el método de identificación de coeficientes.
En un primer paso, vamos a escribirlo como producto de dos polinomios de grado~2 en~$\mathbb{R}$.
\[
x^4-3x^2-6x-2=(x^2+Ax+B)(x^2+Cx+\mathit{D}),\qquad
A,B,C,D\in\mathbb{R}
\]
Para determinar los números $A$, $B$, $C$ y $\mathit D$, utilizamos identificación de coeficientes y, para ello, expandimos el producto del lado derecho.
\[
x^4-3x^2-6x-2=
x^4+(A+C)x^3+(B+AC+\mathit{D})x^2+(A\mathit{D}+BC)x+B\mathit{D}
\]
Dado que dos polinomios son iguales si y solo si los coeficientes correspondientes a los términos del mismo grado son iguales, podemos construir el siguiente sistema de ecuaciones:
\begin{align*}
A+C&=0\\
B+AC+\mathit{D}&=-3\\
A\mathit{D}+BC& =-6\\
B\mathit{D}&=-2
\end{align*}
Este sistema se puede resolver porque proviene de un polinomio sin término de tercer grado,
pero para ello, conviene seguir el camino que indicamos a continuación.
\begin{enumerate}
\item
De la primera ecuación deducimos que $C=-A$; sustituimos $C$ por $-A$ en las dos ecuaciones centrales y obtenemos
\begin{align*}
\mathit{D}+B &=-3+A^2\\
\mathit{D}-B &=\frac{-6}A
\end{align*}
Sumando las dos ecuaciones, podemos expresar $D$ en función de $A$, y restándolas, podemos expresar $B$ en función de $A$:
\[
\mathit{D}=\frac12\Big(-3+A^2-\frac6A\Big),\qquad\qquad B=\frac12\Big(-3+A^2+\frac6A\Big)
\]
\item
A continuación, utilizamos la última ecuación del sistema inicial, $B\mathit{D}=-2$; sustituyendo $B$ y $\mathit{D}$ por las expresiones obtenidas en el punto anterior, obtenemos una ecuación en $A$:
\begin{multline*}
-2=B\mathit{D}= \frac14\Big((-3+A^2)-\frac6A\Big)\Big((-3+A^2)+\frac6A\Big)=\\
=\frac14\Big((-3+A^2)^2-\frac{36}{A^2}\Big) =
\frac14\Big(9-6A^2+A^4-\frac{36}{A^2}\Big)
\end{multline*}
Multiplicando por $4A^2$ en los dos extremos de la igualdad anterior
\[
-8A^2=9A^2-6A^4+A^6-36
\]
Y reagrupando los términos en las potencias de $A$, obtenemos la siguiente ecuación polinómica
\[
A^6-6A^4+17A^2-36=0
\]
Aparentemente, el problema se ha complicado, ya que pasamos de una ecuación de grado 4 a una ecuación de grado 6.
Sin embargo, observamos que en esta última ecuación, todas las potencias de $A$ son pares, por lo que podemos hacer un cambio de variable, $A^2=z$, obteniendo una ecuación de grado 3.
\[
z^3-6z^2+17z-36=0
\]
Utilizando el algoritmo de Ruffini, buscamos una solución entre los divisores de 36, y la encontramos en $z=4$
\begin{center}
\begin{tabular}{r|rrrr}
  & $1$ & $-6$ & $17$ & $-36$ \\
$4$ &   &  $4$ & $-8$ &  $36$\\\hline
  & $1$ & $-2$ &  $9$ & \multicolumn{1}{|r}{$0$}\\\cline{5-5}
\end{tabular}
\end{center}
Dado que la factorización de un polinomio es única, basta tomar una de las soluciones en $A$, ya que cualquier otra solución nos conducirá a la misma factorización.
En este caso, una posible solución verifica $A^2=4$, y podemos considerar $A=2$.
\item
A partir del valor de $A$, ya podemos calcular el valor del resto de los parámetros:
\begin{align*}
C & = -A = -2 \\
B & =\frac12\Big(-3+A^2+\frac6A\Big) = \frac12(-3+4-3) = 2\\
\mathit{D} & =\frac12\Big(-3+A^2-\frac6A\Big) = \frac12(-3+4+3)=-1
\end{align*}
Por lo tanto:\quad $x^4-3x^2-6x-2=(x^2+2x+2)(x^2-2x-1)$
\end{enumerate}
%
Todavía no podemos concluir que esta sea la factorización en $\mathbb{R}$, antes debemos comprobar si los dos polinomio son irreducibles.
\begin{align*}
x^2+2x+2=0 &\quad \Longrightarrow\quad x= \frac{-2\pm\sqrt{4-8}}2=-1\pm\mathrm{i}\\
x^2-2x-1=0 &\quad \Longrightarrow\quad x= \frac{2\pm\sqrt{4+4}}2=1\pm\sqrt2
\end{align*}
Por lo tanto, el primer polinomio de grado 2 es irreducible en $\mathbb{R}$, pero  el segundo no lo es.
En cualquier caso, ya podemos escribir las dos factorizaciones:

\noindent
Factorización en $\mathbb{R}$:
\[
x^4-3x^2-6x-2=(x^2+2x+2)(x-1-\sqrt2)(x-1+\sqrt2)
\]
%
Factorización en $\mathbb{C}$:
\[
x^4-3x^2-6x-2=(x+1-\mathrm{i})(x+1+\mathrm{i})(x-1-\sqrt2)(x-1+\sqrt2)\fejeq
\]
\end{ejemplo}


\begin{ejemplo}\label{ej:factorR}
%:ej:factorR
En este ejemplo, vamos a factorizar el polinomio $x^4+4$ en $\mathbb{R}$ siguiendo el método del ejemplo anterior.
\[
x^4+4=(x^2+Ax+B)(x^2+Cx+\mathit{D}),\qquad
A,B,C,D\in\mathbb{R}
\]
Expandiendo el lado derecho de la igualdad y agrupando los términos obtenemos
\[
x^4+4=
x^4+(A+C)x^3+(B+AC+\mathit{D})x^2+(A\mathit{D}+BC)x+B\mathit{D}.
\]
Identificamos coeficientes y obtenemos el siguiente sistema:
\begin{align*}
A+C&=0\\
B+AC+\mathit{D}&=0\\
A\mathit{D}+BC&=0\\
B\mathit{D}&=4
\end{align*}
De la primera ecuación deducimos que $C=-A$, y sustituyendo $C$ por $-A$ en las dos ecuaciones centrales obtenemos:
\begin{align*}
\mathit{D}+B&=A^2\\
A(\mathit{D}-B)&=0
\end{align*}
Necesariamente, $A\ne 0$, ya que en caso contrario, el sistema se reduciría a $B+D=0$, $BD=4$, y de ahí: $-B^2=4$, lo que es imposible.
Por lo tanto, las ecuaciones quedan:
\begin{align*}
\mathit{D}+B&=A^2\\
\mathit{D}-B &=0;
\end{align*}
y sumándolas y restándolas, podemos escribir $B$ y $\mathit D$ en función de $A$: 
\[
\mathit{D} = \dfrac{A^2}2,\qquad
B = \dfrac{A^2}2.
\]
Sustituyendo en la última ecuación del sistema inicial, $B\mathit{D}=4$:
\begin{align*}
\dfrac{A^2}2\cdot\dfrac{A^2}2 &=4\\
A^4 &=16
%\\
%A &= \pm\sqrt2
\end{align*}
Nos quedamos con la solución $A=2$ y terminamos de calcular el resto de coeficientes
\[
C=-A=-2,\qquad B= \dfrac{A^2}2= 2,\qquad D=\dfrac{A^2}2= 2
\]
y escribimos la factorización obtenida:
\[
x^4+4=(x^2+2x+2)(x^2-2x+2)
\]
En este caso, los dos polinomios de grado 2 son irreducibles, por lo que esa es la factorización en $\mathbb{R}$.\fej
\end{ejemplo}

Dado que todas las magnitudes físicas se pueden medir con números reales, se podría pensar que los números complejos solo son un objeto matemático abstracto sin interés práctico.
Sin embargo, la utilidad de estos números no está en la descripción de magnitudes físicas, sino que constituyen una herramienta para resolver problemas algebraicos y geométricos.
En el ejemplo anterior, hemos factorizado un polinomio en $\mathbb{R}$ usando solamente números reales;
en el siguiente ejemplo, vamos a resolver el mismo ejercicio pero ayudándonos de los números complejos.
%
\begin{ejemplo}\label{ej:factor}
Vamos a factorizar el polinomio $P(x)=x^4+4$ en $\mathbb{C}$ y en $\mathbb{R}$.
Introduciendo números complejos, podemos realizar fácilmente la siguiente factorización:
\[
x^4+4=x^4-(2\mathrm{i})^2=(x^2-2\mathrm{i})(x^2+2\mathrm{i}).
\]
Para seguir factorizando, resolvemos las ecuaciones $x^2-2\mathrm{i}=0$ y $x^2+2\mathrm{i}=0$.
Para la primera, buscamos $x=a+b\mathrm{i}$, con $a,b\in\mathbb{R}$, tal que $(a+b\mathrm{i})^2=2\mathrm{i}$, es decir,
\begin{align*}
%(a+b\mathrm{i})^2&= 2\mathrm{i}\\
a^2-b^2+2ab\mathrm{i} &= 2\mathrm{i}
\end{align*}
A partir de esta igualdad, comparando las partes reales e imaginarias, construimos el siguiente sistemas de ecuaciones en $\mathbb{R}$:
\[
a^2-b^2 = 0,\qquad ab = 1,
\]
cuyas soluciones son $\{a_1=1,b_1=1\}$, $\{a_2=-1,b_2=-1\}$;
es decir, las soluciones de $x^2-2\mathrm{i}=0$ son
\[
1+\mathrm{i},\qquad
-1-\mathrm{i}
\]
Siguiendo el mismo método, obtenemos las soluciones de $x^2+2\mathrm{i}=0$:
\[
1-\mathrm{i},\qquad
-1+\mathrm{i}
\]
En consecuencia, la factorización en $\mathbb{C}$ del polinomio $x^4+1$ es
\[
x^4+4=(x-1-\mathrm{i})(x+1+\mathrm{i})(x-1+\mathrm{i})(x+1-\mathrm{i})
\]
%
Para obtener la factorización en $\mathbb{R}$ basta multiplicar los factores correspondientes a las raíces conjugadas.
De esa forma, la identidad $(A+B)(A-B)=A^2-B^2$ elimina la unidad imaginaria.
\begin{align*}
x^4+4 &=\big((x+1)-\mathrm{i}\big)\big((x+1)+\mathrm{i}\big)
\big((x-1)-\mathrm{i}\big)\big((x-1)+\mathrm{i}\big)=\\
&=\big((x+1)^2+1\big)\big((x-1)^2+1\big)=\\
&=(x^2+2x+2)(x^2-2x+2)\fejeq
\end{align*}
\end{ejemplo}

El esquema seguido en este ejemplo es muy habitual en matemáticas: para resolver un problema en $\mathbb{R}$, lo estudiamos antes en $\mathbb{C}$ para aprovecharnos de las propiedades adicionales;
posteriormente volvemos a $\mathbb{R}$ para dar las soluciones deseadas.
A lo largo del tema veremos más ejemplos de esta metodología.

También nos hemos encontrado en estos dos últimos ejemplos con algo que será muy recurrente en matemáticas: los problemas admiten distintos caminos para llegar a su solución.
Debemos aprender las distintas herramientas y métodos alternativos y saber elegir en cada momento el más adecuado y simple.


\subsubsection{EJERCICIOS:}

\begin{enumerate}

\item
Factorice en $\erre$ y en $\ce$ el polinomio $x^4-2x^2+8x-3$.
¿Cuáles son las soluciones de la ecuación $x^4-2x^2+8x-3=0$?


\item
Factorice en $\erre$ y en $\ce$ los siguientes polinomios:
\setcontadoralph
\vspace{-.5em}
\begin{centrar}
\nitem
$16x^4-56x^2-256x+561$\hfill
\nitem
$4x^4+8x^3+17x^2+2x+4$
%$x^4-6x^3+12x^2-18x+27$
\end{centrar}


\item
En este ejercicio vamos a aprender a factorizar un polinomio de grado 4 con término cúbico.
Consideremos el polinomio $P(x)=x^4-4x^3+9x^2-4x+8$:
%\[
%%P(x)=x^4-x^3+x^2+2
%%P(x)=4x^4+8x^3+17x^2+2x+4
%P(x)=x^4-4x^3+9x^2-4x+8
%\]
\begin{enumerate}
\item
Exprese el polinomio $P$ cambiando su centro a 1 y, mediante el cambio de variable $t=x-1$ conviértalo en otro polinomio $Q(t)$. (Que no debe contener término de grado 3).
\item
Factorice el polinomio $Q(t)$ usando identificación de coeficientes.
%siguiendo el método del ejemplo~\ref{ej:factorR} (página~\pageref{ej:factorR}).
\item
Obtenga la factorización en $\erre$ de $P(x)$ deshaciendo el cambio de variable y finalmente, factorice el polinomio en $\ce$.
\end{enumerate}


\item
En este ejercicio vamos a calcular $\cos\frac{\pi}{5}$ utilizando números complejos.
\begin{enumerate}
\item
Utilice la fórmula de De Moivre para demostrar que
\[\cos(5\theta)=16\cos^5\theta-20\cos^3\theta+5\cos\theta\]
\item
Deduzca que $\cos\frac{\pi}{5}$ es raíz del polinomio $P(x)=16x^5-20x^3+5x+1$
\item
Factorice el polinomio $P(x)$ determinando una raíz ``a ojo'' y usando identificación de coeficientes para factorizar el polinomio de grado 4 resultante como producto de dos polinomios de grado 2.
\item
A partir de la factorización del apartado anterior, calcule $\cos\dfrac{\pi}{5}$
\end{enumerate}


\end{enumerate}